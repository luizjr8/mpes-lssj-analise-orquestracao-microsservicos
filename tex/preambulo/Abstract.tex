\chapter*{Abstract}
This work addresses the challenge of microservices orchestration in building multimodal virtual assistants for the financial sector, a domain that demands high performance, scalability, and resilience. The adoption of decentralized architectures, while offering greater modularity and agility, introduces complexities related to inter-service communication, particularly in processing voice and text data flows.

The research employs a mixed-methods methodology, combining a systematic literature review with a quantitative experimental study. In a prototype application, the communication technologies \gls{rest}, \gls{grpc}, and Thrift were compared under progressively intense load scenarios. The analysis focused on performance and resource efficiency metrics, such as latency (\textit{p50}, \textit{p95}, \textit{p99}), throughput, and CPU/memory usage.

The findings provide technical insights into the trade-offs between integration simplicity (\gls{rest}) and binary/\acrshort{http}2 efficiency (\acrshort{grpc}/Thrift), resulting in practical guidelines for architectural decisions and a replicable evaluation protocol for assistant systems. The contribution lies in the advancement of software engineering applied to intelligent distributed systems, offering empirical evidence and artifacts that support technology choices aligned with the demands of the financial market.

\paragraph{Keywords:} microservices; REST; gRPC; Thrift; virtual assistants; multimodal; software engineering; financial sector.