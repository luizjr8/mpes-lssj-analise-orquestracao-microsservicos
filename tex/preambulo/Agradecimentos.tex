\chapter*{Agradecimentos}
\thispagestyle{empty}

Gostaria de expressar minha profunda gratidão a todas as pessoas e instituições que tornaram possível a realização deste trabalho.

Em primeiro lugar, agradeço a \textbf{Deus} pela luz e pela orientação ao longo desta jornada de aprendizado e crescimento.

À \textbf{\orientadora}, minha orientadora, dedico meus sinceros agradecimentos pela orientação precisa, pelas sugestões de leitura enriquecedoras e pelo acompanhamento constante desde a concepção até a finalização deste estudo. Sua presença foi fundamental em cada etapa do processo.

À minha família, meu eterno agradecimento pelo apoio incondicional. Em especial, à minha esposa \textbf{Elaine Jerônimo}, companheira dedicada em todos os momentos, cuja inteligência e força sempre foram fonte de inspiração. Aos meus amados filhos, \textbf{Luiz Felipe} e \textbf{Maria Fernanda}, pelo amor genuíno que enche meus dias de alegria e significado, e por serem minha maior motivação para buscar sempre o melhor.

Aos meus pais \textbf{Lanúsia Lucena} e \textbf{Luiz Silva}, agradeço por terem sido os alicerces da minha educação e por sempre acreditarem no meu potencial. Sua confiança inabalável e apoio constante foram a base que me permitiu chegar até aqui e conquistar todos os meus objetivos.

Aos professores e funcionários da \textbf{CESAR School}, meu reconhecimento por contribuírem diretamente para minha formação na área de Ciência da Computação, elevando meu conhecimento técnico e minha prática profissional.

À \textbf{Universidade de São Paulo (USP)}, instituição da qual também sou aluno, agradeço pelo acesso aos recursos e ferramentas essenciais para esta pesquisa, bem como pelo ambiente propício à inovação e pelos trabalhos acadêmicos de excelência que serviram como inspiração para este estudo.

Por fim, manifesto minha gratidão à \textbf{Comunidade \textit{Open Source}}, cuja filosofia tem fundamentado o desenvolvimento de \textit{software}. Desde o \textit{Linux} até as diversas ferramentas utilizadas neste trabalho - incluindo o~\LaTeX~para a escrita e os protocolos abertos que foram objeto de estudo - reconheço que são fruto de contribuições coletivas que garantem longevidade e excelência técnica à nossa área.
