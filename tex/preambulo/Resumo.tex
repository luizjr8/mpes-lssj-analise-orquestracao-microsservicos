\chapter*{Resumo}

Este trabalho aborda o desafio da orquestração de microsserviços na construção de assistentes virtuais multimodais para o setor financeiro, um domínio que exige alta performance, escalabilidade e resiliência. A adoção de arquiteturas descentralizadas, embora ofereça maior modularidade e agilidade, introduz complexidades relacionadas à comunicação entre serviços, especialmente no processamento de fluxos de dados de voz e texto.

A pesquisa adota uma metodologia de métodos mistos, combinando uma revisão sistemática da literatura com um estudo experimental quantitativo. Em uma aplicação protótipo, foram comparadas as tecnologias de comunicação \gls{rest}, \gls{grpc} e Thrift sob cenários de carga progressivamente intensos. A análise focou em métricas de desempenho e eficiência de recursos, como latência (\textit{p50}, \textit{p95}, \textit{p99}), throughput, e uso de CPU e memória.

Os achados quantificam subsídios técnicos entre a simplicidade de integração (\gls{rest}) e a eficiência binária/\acrshort{http}2 (\acrshort{grpc}/Thrift), resultando em diretrizes práticas para a decisão arquitetural e em um protocolo de avaliação replicável para sistemas de assistentes. A contribuição posiciona-se no avanço da engenharia de software aplicada a sistemas distribuídos inteligentes, oferecendo evidências e artefatos que apoiam escolhas tecnológicas alinhadas às exigências do mercado financeiro.

\paragraph{Palavras-chave:}microsserviços; REST; gRPC; Thrift; assistentes virtuais; multimodal; engenharia de software; mercado financeiro.
